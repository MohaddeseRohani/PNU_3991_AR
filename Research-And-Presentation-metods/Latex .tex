\documentclass[a4paper,12pt]{article}
\usepackage[top=2cm,right=1cm,bottom=2cm,left=1cm]{geometry} 
\usepackage{hyperref,fontspec}
\usepackage{setspace}
\usepackage{pdfpages}
\doublespacing
%\makeatletter
%\appto{\endmulticols}{\@doendpe}
%\makeatletter
%\fancypage{}{\setlength{\fboxrule}{0.01cm}\ovalbox}
%\setlength\parindent{5pt}
\title{\textbf{Project : Working with Latex}}
\author{\textbf{Mohaddese Rouhani}}
\date{\textbf{13 01 2021}}
\begin{document}
	\maketitle 
	\newpage
	\section{CHALLENGES OF e-RESEARCH COLLABORATION}
	In addition to its potential benefits, the Net does create some new challenges for collaborative research. For example, one application of the Net is to create an environment in which the challenges of distributed research groups and the needs of the community can be met. However, such a substitution of Net-based for real-life contexts presupposes Internet efficacy and access (to hardware,bandwidth,a space,and a time) by all the participants. Such tools or skills are not always found among active researchers even in today's modern universities and are much rarer in higher education institutions and research units in developing countries.\\
	Cyber scholar Mark Nunes (see Nunes,1997) provides an account of his experiences with Internet-based scholarly activities.They include collaborating on writing
	projects,planning conferences,and participating in lists and MOOs. Nunes outlines a
	host of benefits and overviews problems that result from a lack of commitment in
	cyberspace.He writes:\\ \\
	In addition, the lack of physical "space" has a serious impact on the commitment of your online collaborators. We all showed up this afternoon. If four of us agreed to meet at 1:15 for some sort of online panel, I could almost guarantee at least one of us would be missing. In the Postmodern Spacings project, at times real-time discussion consisted of only two people. In fact, we never had more than half of the participants present at any one meeting. In a similar vein, online collaborative projects like Postmoderm Spacings make closure a difficult goal. While the medium is wonderful for new openings,the looseness makes completion quite a task.\\ \\
	One could argue that because Nunes's group used synchronous groups excessively,a lack of commitment stemmed from forcing synchronous activity on busy people, but that would not be the full story. Research collaboration depends on frequent formal and informal conversation to be most effective. Most of us can only cominit a certain amount of time in our lives for such interaction too often the face-to-face one takes priority. The pure technical determinist might argue that an effective project management tool could eliminate wasted time, logistical complexity, and errors due to misunderstanding. The predilection for face-to-face interaction amongst almost all researchers we know constrains effective collaboration-but it usually does not stop the collaborative process. As we gain more experience designing and completing collaborative e-research projects,
	and the tools continue to improve, this barrier will be reduced, though never eliminated. As in all life/space decisions, trade-offs between convenience, cost, effectiveness, and time commitment are a component of the collaborative research process.\\
	\\
	\section{COLLABORATION TOOLS IN ACTION: A FAILED\\ EXAMPLE}
	Most of the examples in this hook document successful applications of e-research tools and techniques. The following example illustrates that success is not an assured outcome. In this example we describe the context and the tools ofa collaborative e-research project.This research team was funded for a two-year project to investigate and develop educational applications for next-generation, high-speed Internet networks. The researchers were located at seven Canadian universities, and their work focused on developing Net-based video conference applications, creating repositories for educational objects,and repurposing of educational video content for distribution over the Net. The project administrative team was located in one of the universities.\\
	while a single researcher or small team was located at the other universities. In early discussions, the researchers noted the need for asynchronous communication, file, and project management tools. They selected a free Net-based product (CommunityZero) that provided these and a variety of additional tools (such as news features, synchronous chats, calendars, etc.) that could be used by the group.One of the principle investigators in the project set up a demonstration in which the original project application. a few updated notices, and an asynehronous discussion were established as a prototype of the community collaboration tool's potential.This prototype was demonstrated at one of the few face-to-face meetings of the group,and all agreed to use the system.However, many of the collaborators did not sign on to the system, nor did they respond to the application manager inviting them to do so. Soon, those who had signed on noticed that the asynchronous discussion had little new material and that no one was adding new content to the collaborative workspace. The site failed to achieve critical mass and within six months was removed hy the Web site owners.\\
	Why did the system not prove beneficial? Despite the need for collaboration tools, these application tools were perceived as less valuable than more readily accessible and familiar tools.Most of the administrative staff, who were located at a single site,had access to a variety of LAN-based services. Most of the remote participants were less involved in the project and thus did not interact with the collaborative team software on a daily basis. This particular project was just one of many workplace obligations for the remote and distributed group. Although the software had the capacity for the owner to push e-mail announcements to participants, there was no way to automatically alert all members when new content was added. T'hus, users logged on to a remote system only to find that there was nothing new in most instances.Such negative feedback quickly extinguishes the desire to log on to external sites. After a few months of nonuse, many participants forget passwords and login I.D.s, making further participation impossible without negotiating with systems operators.\\
	Thus, a collaborative system like any new intervention must, first. add a relative
	advantage (Rogers,1995), second, be compatible with the current workload of all participants,and. third, require minimal effort to participate. The availability of more compatible means of communication (email and FTP sites) plus the lack of incentives for participants to log in on a frequent basis meant that the software did not add significant value to the project and thus was abandoned.\\
	\\
	\section{APPLICATIONS OF COLLABORATIVE SOFTWARE BY e-RESEARCHERS}
	As the example above demonstrates, the need for communication and collaboration
	permeates many aspects of e-research. This is most obvious when there is a team of
	researchers. but collaboration is also useful for the solitary researcher during those components of the research process when results, discussions, or questions must be shared with others.\\
	Net-based collaborative software is relatively new, and we are just discovering ways that e-research teams can use these powerful tools. As noted in earlier chapters, the e-research process commences with selection and refinement of a research problem. 	Often this process is iterative, as various drafts are shared with sponsors, supervisors, or members of the research team. Placing initial drafts in a Net-accessible file space allows for controlled access to important documents. During the literature review process, a collaborative workspace is also useful for documenting, summarizing, and sharing insights from research. Each of the three software packages reviewed later in this section provides space for listing and annotating Web sites, databases, and other references found on the Net.The polling feature of these collaborative packages is handy for conducting quick surveys of opinions and priorities of team members. The calendar feature is useful for setting deadlines and for reminding members of real-time meetings and consultations. During the data collection phase of the research, the calendar is also useful for scheduling interviews and for circulating and archiving drafts of survey instruments or coding protocols. An important feature of much qualitative research is the memoing feature (Nunes, 1997), by which researchers document ideas, insights, questions,and observations during the data collection and analysis phases of research. Bogdan and Biklen (1982) extend the role of these memos in research to"the mainstay of qualitative research . . . a written account of what the researcher hears, sees,experiences,and thinks in the course of collecting and reflecting on the data in a qualitative study" (p. 74).\\
		Quantitative data sets can also be shared in common workspace so individual
	members can run tests on the data without bothering other team members for access.
	Finally, in the dissemination stage, opinions and suggestions can be polled and final drafts of results and “to do" lists can be shared amongst members. Although they are not necessarily Web-based, Microsoft Word's editing tools provided convenient ways to suggests edits and add comments as we composed this book. These tools are also very useful when marking student papers and exercises. Considerable time can be saved when using Microsoft's editing tools by displaying the “reviewing" toolbar from the “view" and “toolbars" options.The review toolbar allows one to quickly add comments and edits, accept or reject the edits of coauthors, and turm the edit feature on and off.\\
	There are a number of software packages on the market that are designed to support these functions. The first generation of these products creats a shared workspace
	on a central. Net-connected server. Individual users log on to access and to add to
	these central services. More recently. software has been designed to use the computing power of remote users more extensively and to allow team members to share
	resources and communicate through “peer-to-peer" technologies. The most infamous
	of these peer-to-peer packages is Napster, which revolutionized how commercial
	music is distributed. We review three of these products, not to endorse them, but to illustrate how these tools can be used to enhance the e-research process.\\
	\\
	\section{MICROSOFT'S SHAREPOINT\textsuperscript{TM} TEAM SERVICES}
	SharePoint Team Services were introduced as components of the Microsoft Office,
	version XP and were also included with later versions of Front Page. ...
\end{document}
